\documentclass{book}

% ======= Use packages ===

% Breaking huge tables to prevent overflow
\usepackage{tabularx}
% Adding toprule and bottomrule
\usepackage{booktabs}

\begin{document}

% =========== Main Part ========

% Horizontal table example
\begin{table}

	% Caption written in title case
	\caption{Top 10 Importing and Exporting Countries of ICT Goods 2017}
		
	\centering
	
	\begin{tabular}{ccccccc} % each 'c' indicates one column

	% Toprule as a thicker line to be added
	\toprule
	
			&	\textbf{Top 10 exporters} &			&&			&	\textbf{Top 10 importers} &\\
			
	% Horizontal line ranging from column 1 until column 3	and another line ranging from column 5 up to column 7
	\cline{1-3} \cline{5-7}
	
	
	Rank	&	Country				&	USD billion	&&	Rank	&	Country				& USD billion \\
	
	\cline{1-3} \cline{5-7}
	
	
	1		&	China				&	6.13		&&	1		&	USA					& 3.44 \\
	2		&	Korea				&	1.42		&&	2		&	China				& 3.30 \\
	3		&	Taiwan				&	1.39		&&	3		&	Hong Kong			& 3.06 \\
	4		&	Singapore			&	1.20		&&	4		&	Germany				& 1.02 \\
	5		&	Germany				&	0.72		&&	5		&	Singapore			& 0.91 \\
	6		&	USA					&	0.69		&&	6		&	Japan				& 0.87 \\
	7		&	Malaysia			&	0.67		&&	7		&	Korea				& 0.72 \\
	8		&	Mexico				&	0.66		&&	8		&	Taiwan				& 0.64 \\
	9		&	Japan				&	0.58		&&	9		&	Mexico				& 0.63 \\
	10		&	Netherlands			&	0.56		&&	10		&	Netherlands			& 0.63 \\
	
	% Adding a thicker line at the bottom
	\bottomrule
	
	
	\end{tabular}
	
	% Adding the sources
	\begin{flushleft}
	\textit{Note.} Adapted from Source (2019)
	\end{flushleft}
	
	% Labeling the table for referencing it in the text
	\label{tab:top10_importers_exporters}
	
\end{table}


\end{document}
